\documentclass[10pt, letterpaper]{article}

\usepackage [english]{babel}

\title{Rootkit of gruppe 6}
\author{Roman Karlstetter \and Philipp M\"uller}
\date{\today}

\begin{document}

\maketitle

\section{Overview}

This rootkit implements several functions, that can -- independently of each other -- be turned on and off. They are outlined here:

\begin{description}
\item [File hiding] It is possible to prevent some files to be shown. Files beginning with a certain prefix are hidden from e.g. \texttt{ls}. However, these files stay accessible.
\item [Process hiding] It is possible to hide certain processes by their ID.
\item [Module hiding] It is possible to hide the rootkit module from \texttt{lsmod}.
\item [Socket hiding] The rootkit can hide certain TCP and UDP sockets from \texttt{netstat} and \texttt{ss}.
\item [Privilege escalation]
\item [Covert communication]
\end{description}

\section{Functions of the rootkit}

\end{document}
